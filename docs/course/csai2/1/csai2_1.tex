% recipes: xelatex
\documentclass{article}
\usepackage{amsmath}
\usepackage{amssymb}
\usepackage{amsthm}
\usepackage{amsfonts}
\usepackage{ctex}

\begin{document}

\section{引论}

\subsection{向量和矩阵范数}

\subsubsection{向量范数的定义}

称$\mathbb{R}^n$上的一个函数$\Vert\cdot\Vert$为范数,若满足:
\begin{enumerate}
  \item \textbf{非负性}\quad $\Vert\mathbf{x}\Vert\ge0$, \quad $\Vert\mathbf{x}\Vert=0$当且仅当$\mathbf{x}=\mathbf{0}$
  \item \textbf{齐次性}\quad $\Vert\alpha\mathbf{x}\Vert=|\alpha|\Vert\mathbf{x}\Vert, \ \alpha\in\mathbb{R}$
  \item \textbf{三角不等式}\quad $\Vert\mathbf{x}+\mathbf{y}\Vert\le\Vert\mathbf{x}\Vert+\Vert\mathbf{y}\Vert$
\end{enumerate}

常见的向量范数有:

\begin{itemize}
  \item \textbf{1-范数}\quad $\Vert\mathbf{x}\Vert_{1}=\sum_{i=1}^{n}|x_{i}|$
  \item \textbf{2-范数}\quad $\Vert\mathbf{x}\Vert_{2}=\left(\sum_{i=1}^{n}|x_{i}|^{2}\right)^{\frac 1{2}}$
  \item \textbf{$\infty$-范数}\quad $\Vert\mathbf{x}\Vert_{\infty}=\max\limits_{1\le i\le n}|x_{i}|$
\end{itemize}

\subsubsection{向量范数的性质}

\paragraph{向量范数的等价性}
对于$\mathbb{R}^n$中的任意两种范数$\Vert\mathbf{x}\Vert_{\alpha}$和$\Vert\mathbf{x}\Vert_{\beta}$,都存在两个正数$c_1,c_2$,使得对任意$\mathbf{x}\in\mathbb{R}^n$都有
$$c_1\Vert\mathbf{x}\Vert_{\alpha}\le\Vert\mathbf{x}\Vert_{\beta}\le c_2\Vert\mathbf{x}\Vert_{\alpha}$$

\begin{proof}

令$f(\mathbf{x})=\Vert\mathbf{x}\Vert_{\beta}$,$\mathbf{S}=\{\mathbf{x}\in\mathbb{R}^n: \Vert\mathbf{x}\Vert_{\alpha}= 1\}$为一个有界闭集,
则连续函数$f(\mathbf{x})$在$\mathbf{S}$上存在最小值$c_1=\min\limits_{\mathbf{x}\in\mathbf{S}}f(\mathbf{x})$和最大值$c_2=\max\limits_{\mathbf{x}\in\mathbf{S}}f(\mathbf{x})$。
对于$\mathbf{x}\ne \mathbf{0}$有$\frac{\mathbf{x}}{\Vert\mathbf{x}\Vert_{\alpha}}\in \mathbf{S}$,则有
$$c_1\le f(\frac{\mathbf{x}}{\Vert\mathbf{x}\Vert_{\alpha}})=\frac{\Vert\mathbf{x}\Vert_{\beta}}{\Vert\mathbf{x}\Vert_{\alpha}}\le c_2$$
即$$c_1\Vert\mathbf{x}\Vert_{\alpha}\le\Vert\mathbf{x}\Vert_{\beta}\le c_2\Vert\mathbf{x}\Vert_{\alpha}$$

\end{proof}

对于常用的向量范数,有如下关系:
$$\Vert\mathbf{x}\Vert_2\le\Vert\mathbf{x}\Vert_1\le\sqrt{n}\Vert\mathbf{x}\Vert_2$$
$$\Vert\mathbf{x}\Vert_{\infty}\le\Vert\mathbf{x}\Vert_1\le n\Vert\mathbf{x}\Vert_{\infty}$$
$$\Vert\mathbf{x}\Vert_{\infty}\le\Vert\mathbf{x}\Vert_2\le\sqrt{n}\Vert\mathbf{x}\Vert_{\infty}$$

\paragraph{向量序列的收敛性}
在空间$\mathbb{R}^n$中,向量序列$\{\mathbf{x}^{(k)}\}$收敛于向量$\mathbf{x}^*$的充要条件是存在范数$\Vert\cdot\Vert$使得
$$\lim_{k\to\infty}\Vert\mathbf{x}^{(k)}-\mathbf{x}^*\Vert=0$$

\paragraph{压缩映射}
设有非空集合$\mathbf{D}\subset\mathbb{R}^n$,对于映射$\mathbf{f}:\mathbf{D}\to\mathbf{D}$,若存在范数$\Vert\cdot\Vert$和常数$q\in[0,1)$使得对任意$\mathbf{x},\mathbf{y}\in\mathbf{D}$都有
$$\Vert \mathbf{f}(\mathbf{x})-\mathbf{f}(\mathbf{y})\Vert\le q\Vert \mathbf{x}-\mathbf{y}\Vert$$
则称$\mathbf{f}$为$\mathbf{D}$上的压缩映射。

\paragraph{Banach压缩映射原理}
设$\mathbf{D}\subset\mathbb{R}^n$为闭集,映射$\mathbf{f}$为$\mathbf{D}$上的压缩映射,则$\mathbf{f}$在$\mathbf{D}$上有唯一不动点$\mathbf{x}$,使得$\mathbf{f}(\mathbf{x})=\mathbf{x}$。

\begin{proof}
设$\mathbf{x}^{(0)}\in\mathbf{D}$,构造序列$\{\mathbf{x}^{(k)}\}$如下:
$$\mathbf{x}^{(k+1)}=\mathbf{f}(\mathbf{x}^{(k)}),\quad k=0,1,2,\ldots$$
则对任意$k\ge0$,有
$$\Vert\mathbf{x}^{(k+1)}-\mathbf{x}^{(k)}\Vert=\Vert\mathbf{f}(\mathbf{x}^{(k)})-\mathbf{f}(\mathbf{x}^{(k-1)})\Vert\le q\Vert\mathbf{x}^{(k)}-\mathbf{x}^{(k-1)}\Vert$$
由此可得
$$\Vert\mathbf{x}^{(k+1)}-\mathbf{x}^{(k)}\Vert\le q^k\Vert\mathbf{x}^{(1)}-\mathbf{x}^{(0)}\Vert$$
对于$m>n\ge0$,有
$$\Vert\mathbf{x}^{(m)}-\mathbf{x}^{(n)}\Vert\le\sum_{k=n}^{m-1}\Vert\mathbf{x}^{(k+1)}-\mathbf{x}^{(k)}\Vert\le\sum_{k=n}^{m-1}q^k\Vert\mathbf{x}^{(1)}-\mathbf{x}^{(0)}\Vert=\frac{q^n-q^m}{1-q}\Vert\mathbf{x}^{(1)}-\mathbf{x}^{(0)}\Vert$$
由于$q\in[0,1)$,当$n\to\infty$时
$$\lim_{n\to\infty}\Vert\mathbf{x}^{(m)}-\mathbf{x}^{(n)}\Vert=0$$
即序列$\{\mathbf{x}^{(k)}\}$为Cauchy序列,故在$\mathbb{R}^n$中收敛,设其极限为$\mathbf{x}^*\in\mathbf{D}$,则由映射的连续性可得
$$\mathbf{x}^*=\lim_{k\to\infty}\mathbf{x}^{(k+1)}=\lim_{k\to\infty}\mathbf{f}(\mathbf{x}^{(k)})=\mathbf{f}(\lim_{k\to\infty}\mathbf{x}^{(k)})=\mathbf{f}(\mathbf{x}^*)$$
即$\mathbf{x}^*$为$\mathbf{f}$的不动点。
设存在不动点$\mathbf{x}_1,\mathbf{x}_2$,则有
$$\Vert\mathbf{x}_1-\mathbf{x}_2\Vert=\Vert\mathbf{f}(\mathbf{x}_1)-\mathbf{f}(\mathbf{x}_2)\Vert\le q\Vert\mathbf{x}_1-\mathbf{x}_2\Vert$$
由于$q\in[0,1)$,上式只在$\mathbf{x}_1=\mathbf{x}_2$时成立,故不动点唯一。

\end{proof}

\subsubsection{矩阵范数的定义}

称$\mathbb{R}^{n\times n}$上的一个函数$\Vert\cdot\Vert$为范数,若满足:
\begin{enumerate}
  \item \textbf{非负性}\quad $\Vert\mathbf{A}\Vert\ge0$, \quad $\Vert\mathbf{A}\Vert=0$当且仅当$\mathbf{A}=\mathbf{0}$
  \item \textbf{齐次性}\quad $\Vert\alpha\mathbf{A}\Vert=|\alpha|\Vert\mathbf{A}\Vert, \ \alpha\in\mathbb{R}$
  \item \textbf{三角不等式}\quad $\Vert\mathbf{A}+\mathbf{B}\Vert\le\Vert\mathbf{A}\Vert+\Vert\mathbf{B}\Vert$
  \item \textbf{矩阵乘法不等式}\quad $\Vert\mathbf{A}\mathbf{B}\Vert\le\Vert\mathbf{A}\Vert\cdot\Vert\mathbf{B}\Vert$
\end{enumerate}

常见的矩阵范数有:

\begin{itemize}
  \item \textbf{列范数}\quad $\Vert\mathbf{A}\Vert_{1}=\max\limits_{1\le j\le n}\sum_{i=1}^{n}|a_{ij}|$
  \item \textbf{行范数}\quad $\Vert\mathbf{A}\Vert_{\infty}=\max\limits_{1\le i\le n}\sum_{j=1}^{n}|a_{ij}|$
  \item \textbf{谱范数}\quad $\Vert\mathbf{A}\Vert_{2}=\sqrt{\lambda_{\max}(\mathbf{A}^T\mathbf{A})}$
  \item \textbf{F-范数}\quad $\Vert\mathbf{A}\Vert_{F}=\sqrt{\sum_{i=1}^{n}\sum_{j=1}^{n}|a_{ij}|^2}$
\end{itemize}

\subsubsection{矩阵范数的性质}

\paragraph{算子范数}
称向量范数导出的矩阵范数为算子范数,定义如下:
$$\Vert\mathbf{A}\Vert=\max_{\mathbf{x}\ne \mathbf{0}}\frac{\Vert\mathbf{A}\mathbf{x}\Vert}{\Vert\mathbf{x}\Vert}=\max_{\Vert\mathbf{x}\Vert=1}\Vert\mathbf{A}\mathbf{x}\Vert$$
由定义可知算子范数是矩阵范数,且与向量范数相容:
$$\Vert\mathbf{A}\mathbf{x}\Vert\le\Vert\mathbf{A}\Vert\cdot\Vert\mathbf{x}\Vert$$

\paragraph{矩阵范数的等价性}

对于$\mathbb{R}^{n\times n}$中的任意两种范数$\Vert\mathbf{A}\Vert_{\alpha}$和$\Vert\mathbf{A}\Vert_{\beta}$,都存在两个正数$c_1,c_2$,
使得对任意$\mathbf{A}\in\mathbb{R}^{n\times n}$都有
$$c_1\Vert\mathbf{A}\Vert_{\alpha}\le\Vert\mathbf{A}\Vert_{\beta}\le c_2\Vert\mathbf{A}\Vert_{\alpha}$$

对于常用的矩阵范数,有如下关系:

\begin{enumerate}
  \item $$\frac 1n\Vert\mathbf{A}\Vert_{\infty}\le\Vert\mathbf{A}\Vert_{1}\le n\Vert\mathbf{A}\Vert_{\infty}$$
  \begin{proof}
  对于任意$a_{ij}\in\mathbf{A}$,有$$|a_{ij}|\le\Vert\mathbf{A}\Vert_{\infty}$$
  则有$$\sum_{i=1}^{n}|a_{ij}|\le n\Vert\mathbf{A}\Vert_{\infty}$$
  取最大值可得$$\Vert\mathbf{A}\Vert_{1}=\max_{1\le j\le n}\sum_{i=1}^{n}|a_{ij}|\le n\Vert\mathbf{A}\Vert_{\infty}$$
  同理可得$$\Vert\mathbf{A}\Vert_{\infty}=\max_{1\le i\le n}\sum_{j=1}^{n}|a_{ij}|\le n\Vert\mathbf{A}\Vert_{1}$$
  \end{proof}
  
  \item $$\frac{1}{\sqrt{n}}\Vert\mathbf{A}\Vert_{\infty}\le\Vert\mathbf{A}\Vert_{2}\le\sqrt{n}\Vert\mathbf{A}\Vert_{\infty}$$
  \begin{proof}
  先证明$\Vert\mathbf{x}\Vert_{\infty}\le\Vert\mathbf{x}\Vert_{2}\le\sqrt{n}\Vert\mathbf{x}\Vert_{\infty}$:
  
  对于任意$x_i\in\mathbf{x}$,有$|x_i|\le\Vert\mathbf{x}\Vert_{2}$,取最大值可得$\Vert\mathbf{x}\Vert_{\infty}\le\Vert\mathbf{x}\Vert_{2}$
  
  对于任意$x_i\in\mathbf{x}$,有$|x_i|\le\Vert\mathbf{x}\Vert_{\infty}$,则有$|x_i|^2\le\Vert\mathbf{x}\Vert_{\infty}^2$,对所有$i$求和可得$\Vert\mathbf{x}\Vert_{2}^2\le n\Vert\mathbf{x}\Vert_{\infty}^2$,即$\Vert\mathbf{x}\Vert_{2}\le\sqrt{n}\Vert\mathbf{x}\Vert_{\infty}$

  由算子范数定义可得
  $$\Vert\mathbf{A}\Vert_{2}=\max_{\mathbf{x}\ne \mathbf{0}}\frac{\Vert\mathbf{A}\mathbf{x}\Vert_{2}}{\Vert\mathbf{x}\Vert_{2}}, \; \Vert\mathbf{A}\Vert_{\infty}=\max_{\mathbf{x}\ne \mathbf{0}}\frac{\Vert\mathbf{A}\mathbf{x}\Vert_\infty}{\Vert\mathbf{x}\Vert_{\infty}}$$

  对于任意$\mathbf{x}\ne \mathbf{0}$,有
  $$\frac{\Vert\mathbf{A}\mathbf{x}\Vert_2}{\Vert\mathbf{x}\Vert_2}\le\frac{\sqrt{n}\Vert\mathbf{A}\mathbf{x}\Vert_\infty}{\Vert\mathbf{x}\Vert_2}\le\frac{\sqrt{n}\Vert\mathbf{A}\mathbf{x}\Vert_\infty}{\Vert\mathbf{x}\Vert_\infty}\le\sqrt{n}\Vert\mathbf{A}\Vert_{\infty}$$
  取最大值可得$$\Vert\mathbf{A}\Vert_{2}\le\sqrt{n}\Vert\mathbf{A}\Vert_{\infty}$$

  对于任意$\mathbf{x}\ne \mathbf{0}$,有
  $$\frac{\Vert\mathbf{A}\mathbf{x}\Vert_\infty}{\Vert\mathbf{x}\Vert_\infty}\le\frac{\Vert\mathbf{A}\mathbf{x}\Vert_2}{\Vert\mathbf{x}\Vert_\infty}\le\sqrt{n}\frac{\Vert\mathbf{A}\mathbf{x}\Vert_2}{\Vert\mathbf{x}\Vert_2}\le\sqrt{n}\Vert\mathbf{A}\Vert_{2}$$
  取最大值可得$$\Vert\mathbf{A}\Vert_{\infty}\le\sqrt{n}\Vert\mathbf{A}\Vert_{2}$$
  \end{proof}
\end{enumerate}

\subsection{误差}

\subsubsection{误差的类型}

误差描述了数值计算中近似解的精确程度,可分为以下几类:
\begin{itemize}
  \item \textbf{截断误差}\quad 在数值运算中运用近似方法表示准确数值运算或数量而引起的,也叫方法误差。
  \item \textbf{舍入误差}\quad 由于计算机字长的限制而产生的误差。
  \item 不与数值方法相关的误差,如测量误差等。
\end{itemize}

\subsubsection{误差的度量}

\paragraph{绝对误差}
设$x^*$为精确值,$\tilde{x}$为近似值,则称$\tilde{x}$的绝对误差为
$$E(\tilde{x})=x^*-\tilde{x}$$
绝对误差具有量纲,反映了近似值与精确值之间的差距,但不能很好地反映近似值的精度。

\paragraph{绝对误差极限}
若$\exists\; \delta>0$,使得$|E(\tilde{x})|=|x^*-\tilde{x}|\le\delta$,则称$\delta$为$\tilde{x}$的绝对误差极限。

\paragraph{相对误差}
设$x^*$为精确值,$\tilde{x}$为近似值,则称$\tilde{x}$的相对误差为
$$E_r(\tilde{x})=\frac{x^*-\tilde{x}}{\tilde{x}}\times 100\; \quad(x^*\ne 0)$$
相对误差不具有量纲,能够较好地反映误差的特性及近似值的精度。

\paragraph{相对误差极限}
若$\exists\; \delta_r>0$,使得$\left| E_r(\tilde{x}) \right|=|\frac{x^*-\tilde{x}}{\tilde{x}}|\le\delta_r$,则称$\delta_r$为$\tilde{x}$的相对误差极限。

\subsubsection{有效数字}

如果近似值$\tilde{x}$的误差不超过某位的半个单位,该位数字到$\tilde{x}$的第一位非零数字共有$n$位,那么这$n$位数字称为$\tilde{x}$的有效数字。
$$\tilde{x}=\pm 10^k\times 0.a_1a_2\cdots a_n$$
$|x^*-\tilde{x}|\le\frac 12\times 10^{k-n}$时称$\tilde{x}$是$x^*$的$n$位有效数字。

\paragraph{误差和有效数字的关系}

由相对误差的定义$E_r(\tilde{x})=\frac{E(\tilde{x})}{|\tilde{x}|}$可知$\delta_r(\tilde{x})=\frac{\delta(\tilde{x})}{|\tilde{x}|}$。
有效数字和相对误差限的关系由以下定理给出:

\begin{itemize}
\item 若$\tilde{x}$有$n$位有效数字,则$\left| \frac{x^* - \tilde{x}}{\tilde{x}} \right| \le \frac{1}{2a_1} \times 10^{1-n}$.
\item 若 $\left|\frac{x^* - \tilde{x}}{\tilde{x}} \right| \le \frac{1}{2(a_1+1)} \times 10^{1-n}$,则$\tilde{x}$至少具有$n$位有效数字。
\end{itemize}

\subsubsection{误差的传播}

\paragraph{函数误差的传播}

若$f(x)$在$\tilde{x}$的邻域上可微,由其在$x=\tilde{x}$的泰勒展开式$f(\tilde{x})\approx f(x^*)+f'(x^*)(\tilde{x}-x^*)$近似可得
$$|f(\tilde{x})-f(x^*)|\le |f'(\tilde{x})|\cdot |\tilde{x}-x^*|$$

由此对近似函数$f(\tilde{x})$的误差限和相对误差限分别有如下估计式:

$$\begin{cases}
\delta f(\tilde{x})\le|f'(\tilde{x})|\cdot\delta(\tilde{x}) \\
\delta_r f(\tilde{x})\le\left|\frac{f'(\tilde{x})}{f(\tilde{x})}\right|\cdot\delta(\tilde{x})
\end{cases}$$

\vspace{2em}

对于二元函数,若$f(x,y)$在$(\tilde{x},\tilde{y})$的邻域上可微,由其泰勒展开式$f(x^*,y^*)\approx f(\tilde{x},\tilde{y})+\frac{\partial f(\tilde{x},\tilde{y})}{\partial x}(x^*-\tilde{x})+\frac{\partial f(\tilde{x},\tilde{y})}{\partial y}(y^*-\tilde{y})$近似可得
$$|f(x^*,y^*)-f(\tilde{x},\tilde{y})|\le \left|\frac{\partial f(\tilde{x},\tilde{y})}{\partial x}\right|\cdot|x^*-\tilde{x}|+\left|\frac{\partial f(\tilde{x},\tilde{y})}{\partial y}\right|\cdot|y^*-\tilde{y}|$$

由此对近似函数$f(\tilde{x},\tilde{y})$的误差限和相对误差限分别有如下估计式:

$$\begin{cases}
\delta f(\tilde{x},\tilde{y})\le \left|\frac{\partial f(\tilde{x},\tilde{y})}{\partial x}\right|\cdot\delta(\tilde{x})+\left|\frac{\partial f(\tilde{x},\tilde{y})}{\partial y}\right|\cdot\delta(\tilde{y}) \\
\delta_r f(\tilde{x},\tilde{y})=\left|\frac{\delta f(\tilde{x},\tilde{y})}{f(\tilde{x},\tilde{y})}\right|
\end{cases}$$

\paragraph{算术误差的传播}

将算术运算视为二元函数,可以算出加减乘除运算的误差传播公式:

$$\begin{cases}
\delta(\tilde{x}\pm\tilde{y})\le\delta(\tilde{x})+\delta(\tilde{y}) \\
\delta_r(\tilde{x}\pm\tilde{y})\le\frac{\delta(\tilde{x})+\delta(\tilde{y})}{|\tilde{x}\pm\tilde{y}|}
\end{cases}$$

$$\begin{cases}
\delta(\tilde{x}\tilde{y})\le|\tilde{y}|\delta(\tilde{x})+|\tilde{x}|\delta(\tilde{y}) \\
\delta_r(\tilde{x}\tilde{y})\le\frac{\delta(\tilde{x})}{|\tilde{x}|}+\frac{\delta(\tilde{y})}{|\tilde{y}|} = \delta_r(\tilde{x}) + \delta_r(\tilde{y})
\end{cases}$$

$$\begin{cases}
\delta(\frac{\tilde{x}}{\tilde{y}}) \le \frac{1}{|\tilde{y}|}\delta (\tilde{x})+|\frac{\tilde{x}}{\tilde{y}^2}|\delta(\tilde{y}) \\
\delta_r(\frac{\tilde{x}}{\tilde{y}}) \le \frac{\delta(\tilde{x})}{|\tilde{x}|}+\frac{\delta(\tilde{y})}{|\tilde{y}|}=\delta_r(\tilde{x})+\delta_r(\tilde{y})
\end{cases}$$

\subsection{数值计算原则}

\subsubsection{适定问题}

称一个数学问题是适定的,如果它满足以下三个条件:

\begin{itemize}
  \item \textbf{存在解}
  \item \textbf{解是唯一的}
  \item \textbf{解连续的取决于初边值条件}
\end{itemize}

即适定问题的解满足存在性、唯一性和稳定性三个条件。否则称其为不适定问题。

\subsubsection{数值稳定性}

对于某个数值算法,其稳定性可分为以下几类:

\begin{itemize}
\item 数值不稳定:输入数据的误差在计算过程中不断扩大
\item 条件稳定(相对稳定):算法在一定条件下数值稳定
\item 无条件稳定(绝对稳定):算法在任何条件下都数值稳定
\end{itemize}

\subsubsection{数值计算原则}

在进行数值计算时,应遵循以下原则:

\begin{enumerate}
\item \textbf{避免两个相近数相减}
\item \textbf{避免用绝对值过小的数做除数}
\item \textbf{防止大数吃掉小数}(避免对数量级差异过大的数作加减法)
\end{enumerate}

除了具体运算中的误差规避,还可以从整体算法设计上控制误差:

\begin{itemize}
\item 简化计算步骤,提高计算效率:\textbf{减少计算量和减少误差积累}
\item 使用数值稳定的算法:\textbf{控制误差的传播}
\end{itemize}

\end{document}
